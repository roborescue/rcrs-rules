%%
\documentclass{article}
%%
\usepackage{amsmath}
\usepackage{amssymb}
\usepackage{booktabs}
\usepackage[shortlabels]{enumitem}
\usepackage{float}
\usepackage[a4paper,left=1in,right=1in,top=1in,bottom=1in]{geometry}
\usepackage[hidelinks]{hyperref}
%%
\begin{document}
%%

\title{RoboCup Rescue Simulation League\\
       2019 Infrastructure Competition Rules}
\author{Luis G. Nardin, Farshid Faraji and Masaru Shimizu}
\date{Version 1.0}
%%
\maketitle
%%
\begin{abstract}
%%
This document describes the 2019 RoboCup Rescue Simulation League rules for the Infrastructure competition.
%%
\end{abstract}
%%
\section{Introduction}
\label{sec:introduction}
%%
Natural disasters are major adverse events that cause large-scale economic, human, and environmental losses. They are usually difficult to predict and even more challenging to prevent from happening. These characteristics demand disaster management strategies to be in place to reduce the negative impact of the disaster. Disaster management is usually divided in four phases:
%%
\begin{itemize}
%%
\item \emph{Mitigation} includes the identification of hazards, the assessment of threats, and the implementation of measures to reduce potential losses and damages.
%%
\item \emph{Preparedness} includes the elaboration of plans to deal with disasters and their consequences. This phase aims to have equipment and procedures ready for use when a disaster occurs.
%%
\item \emph{Response} includes the search and rescue tasks executed during and just after a disaster.
%%
\item \emph{Recovery} takes place after the occurrence of a disaster and involves the assistance of affected people and the restoration of the basic services.
%%
\end{itemize}
%%

Efficacious response plans are essential for reducing the negative impacts of natural disasters. Distributed coordination and planning are crucial for conducting the search and rescue tasks during this phase due to the usually non-existence of a centralized coordinator, involvement of multiple and heterogeneous entities, and lack of communication infrastructure.

Currently these hazardous tasks are performed by humans. However, the RoboCup Rescue Leagues aim to promote the use of artificial entities in performing these tasks.

The RoboCup Rescue Simulation League mission is to promote research and development in this socially significant domain at various levels. First, the league aims to promote the development of new computational artifacts that may improve the distributed coordination and planning among these heterogenous artificial entities when performing search and rescue tasks on disaster scenarios. Second, it aims to provide a simulation software able to realistically represent natural disaster scenarios where these computational artifacts can be assessed. Finally, it aims to promote these research and development activities by conducting competitions to stimulate the exchange of ideas and experience among practitioners.

The Infrastructure competition held during the RoboCup is focused mostly on the third objective.
%%
\section{Infrastructure Competition}
\label{sec:infrastructure}
%%
The Infrastructure competition involves the presentation of already existent tools and simulators of disaster management problems in general. The intent is the evaluation of possible enhancements and expansions of the basic RoboCup Rescue Simulation League simulators (Agent Simulation and Virtual Robots) based on the new ideas and concepts proposed in these tools and simulators. The evaluation will be done in a panel and a winner chosen accordingly to a set of factors related to the technical aspects of the tool or simulator and the presentation. The best tool will be selected for further integration with the simulation platform.
%%
\subsection{Ranking}
%%
The score of each team participating in the Infrastructure Competition is given by the sum of the punctuation given by each other team to a set of evaluative factors related to the innovative aspect of the proposal and the presentation's quality and clarity. The team with the highest sum of scores is the winner of the competition. Each member of Organizing or Technical Committee who is not a member of a team may participate in the scoring. Each team should participate and evaluate the infrastructure presentation.
%%
\subsection{Duty of Release}
%%
Teams participating in the Infrastructure competition must release their source code as an open-source project before the competition. Even though a team is accepted to participate in the competition, if it does not release the source code, a complete manual, prepare an easy installation and running script files (for example, install.sh and run.sh) before the beginning of the RoboCup 2019 (before July 2, 2019), will be disqualified. Thus, it will not be evaluated and considered for winning the prize.
%%
\section*{Acknowledgments}
%%
We thanks all the former RoboCup Rescue Simulation League members of the Technical, Organizing, and Executive Committees for their effort in promoting and organizing this competition. Special thanks to Nobuhiro Ito for his dedication to the RoboCup Rescue Simulation League.
%%
\end{document}
%%